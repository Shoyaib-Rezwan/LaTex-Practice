\section{Define Code Evaluation Method}
The insights of our formative study led to the investigation of our \textbf{RQ1} regarding evaluation dimensions and optimization strategies that programmers normally use. As indicated by DG1, we decided to evaluate the C++ code on five key metrics: redundancy, documentation, clarity, time efficiency, and space efficiency. 

\subsection{Data Collection}
The researchers utilized the Performance-Improving Edits (PIE) dataset \cite{shypula2023learning}, which consists of human-programmer optimizations from CodeNet \cite{puri2021codenet} competitive programming tasks. C++ was chosen as the target language because of its importance in performance-critical applications and its compatibility with the Gem5 simulator. Since C++ submissions often focus on fine-grained optimizations, they provide an ideal basis for analyzing time and space efficiency metrics.

To maintain high data quality, the team filtered submissions to include only functionally correct programs that passed all test cases and runtime constraints. Each program was then paired with its improved version for comparative analysis. By using the Gem5 simulator, the researchers normalized execution times to remove inconsistencies in raw runtime data. The final dataset comprises 77,967 program pairs across various problem domains.

\subsection{Data Analysis and Results }

\subsubsection{High-Quality Code Characteristics}
\begin{figure}[!t]
    \centering
    \includegraphics[width=\linewidth]{dimension_distribution.png}
    \caption{Distributions of key metrics: clarity, redundancy, time efficiency, space efficiency, and documentation in PIE dataset.}
    \Description{Distributions of key metrics: clarity, redundancy, time efficiency, space efficiency, and documentation in PIE dataset.}
    \label{fig:dimension_distribution}
\end{figure}
 We began the analysis by describing the general characteristics of these C++ codes and deciding what parameters needed to be used in a five-dimensional framework. Programs in the PIE dataset demonstrate strong performance across all five optimization dimensions (Figure \ref{fig:dimension_distribution}). Most programs exhibit high clarity (fewer than 10 issues per file) and low redundancy (fewer than 5 redundant constructs), indicating well-structured and efficient code. Space and time efficiency metrics cluster around optimal values, while documentation scores fall within moderate ranges (0.1-0.3). 

 \subsubsection{ Multi-Dimensional Code Quality Analysis}
 \afterpage{
    \begin{table}[!t]
    \centering
    \caption{Evaluation methods, formulas, and optimization examples for each dimension. Notation: $perc(.)$ returns a training-set percentile rank in $[0,1]$; $T_{exec}$ is execution time; $M_{usage}$ is peak memory usage; $Reduntant_{lines} counts duplicate/dead lines$. For clarity, $f_{i}$ are style factors with weights $w_{i}(\Sigma_{i}w_{i}=1)$. For documentation, $S_{sim}$ is the cosine similarity between a code segment and its accompanying natural language documentation (e.g., CodeBERT embeddings). The ceiling operator $\lceil . \rceil$ maps the 0-1 value onto the common 1-10 rubric (top 10\% receive 10). }
    \Description{tab:evaluation_methods}
    \label{tab:evaluation_methods}
    \begin{tabularx}{\linewidth}{l >{\raggedright\arraybackslash}p{0.35\linewidth} >{\raggedright\arraybackslash}X >{\raggedright\arraybackslash}X}
        \toprule
        \textbf{Metric} & \textbf{Formula} & \textbf{Evaluation Method} & \textbf{Examples(Original $\to$ Optimized)}\\
        \toprule
        \textbf{Time Efficiency} & $Score=\lceil 10(1-perc(T_{exec})) \rceil$ & Algorithmic complexity and runtime behavior & $pow(x,2)\to x*x$\\ \midrule
        \textbf{Space Efficiency} & $Score=\lceil 10(1-perc(M_{usage})) \rceil$ & Memory footprint and data structure usage  & vector<vector<int>> mat(n, vector<int>(n,0))$\to$ vector<int>(n*n,0)\\ \midrule
        \textbf{Clarity} & $Score=\lceil 10(1-\Sigma_{i}w_{i}perc(f_{i})) \rceil$ & Magic literals, naming consistency, and statement length Descriptive names and magic number avoidance  & int a=0;$\to$ int sum=0; int f(int x);$\to$ int computeSquare(int x)\\ \midrule
        \textbf{Documentation} & $Score=\lceil 10(perc(S_{sim})) \rceil$ & Embedding-based semantic similarity between code and documentation  & // calculate result $\to$ // Computes total revenue from all entries\\ \midrule
        \textbf{Redundancy} & $Score=\lceil 10(1-perc(Redundant_{lines})) \rceil$ & Duplicate code and dead code detection  & for(...) print(x); for(...) print(x);$\to$ for(...) print(x);\\ \bottomrule
    \end{tabularx}
\end{table}
}


Table \ref{tab:evaluation_methods} presents our analysis of five key metrics across the PIE dataset. The results 
reveal distinct patterns: 
\textbf{Clarity:} 58.65\% of files contain fewer than 10 clarity-related issues, indicating generally well-structured code. 
\textbf{Redundancy:} 32.36\% of files show moderate to high levels of redundant constructs, suggesting optimization opportunities. 
\textbf{Space Efficiency:} 19.19\% of files consume significantly more memory than average, highlighting areas for memory optimization. 
\textbf{Documentation:} 6.30\% of files demonstrate high semantic alignment between code and documentation, while 5.60\% show minimal documentation efforts. 
\textbf{Time Efficiency:} 1.41\% of files achieve exceptional performance, representing solutions that excel across all metrics. 